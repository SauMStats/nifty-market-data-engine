\documentclass[12pt,a4paper]{article}

% ─── Packages ────────────────────────────────────────────────────────────────
\usepackage[a4paper, margin=2.5cm]{geometry}
\usepackage{amsmath, amssymb}
\usepackage{booktabs}
\usepackage{array}
\usepackage{listings}
\usepackage{xcolor}
\usepackage{hyperref}
\usepackage{parskip}
\usepackage{titlesec}
\usepackage{fancyhdr}
\usepackage{microtype}
\usepackage{enumitem}
\usepackage{mdframed}

% ─── Colour palette ──────────────────────────────────────────────────────────
\definecolor{codeblue}{HTML}{1E3A5F}
\definecolor{codegray}{HTML}{F4F6F8}
\definecolor{codered}{HTML}{C0392B}
\definecolor{codegreen}{HTML}{1A7340}
\definecolor{accent}{HTML}{2471A3}
\definecolor{warnyellow}{HTML}{F39C12}

% ─── Listings (code blocks) ───────────────────────────────────────────────────
\lstdefinestyle{python}{
    language=Python,
    backgroundcolor=\color{codegray},
    basicstyle=\ttfamily\small,
    keywordstyle=\color{codeblue}\bfseries,
    stringstyle=\color{codegreen},
    commentstyle=\color{gray}\itshape,
    numberstyle=\tiny\color{gray},
    numbers=left,
    stepnumber=1,
    numbersep=8pt,
    frame=single,
    framerule=0.4pt,
    rulecolor=\color{gray!40},
    breaklines=true,
    breakatwhitespace=true,
    showstringspaces=false,
    tabsize=4,
    xleftmargin=1.5em,
}
\lstset{style=python}

% ─── Info box ─────────────────────────────────────────────────────────────────
\newmdenv[
  backgroundcolor=codegray,
  linecolor=accent,
  linewidth=1.5pt,
  leftmargin=0pt,
  rightmargin=0pt,
  innerleftmargin=10pt,
  innerrightmargin=10pt,
  innertopmargin=8pt,
  innerbottommargin=8pt,
]{infobox}

% ─── Warning box ──────────────────────────────────────────────────────────────
\newmdenv[
  backgroundcolor=yellow!10,
  linecolor=warnyellow,
  linewidth=1.5pt,
  leftmargin=0pt,
  rightmargin=0pt,
  innerleftmargin=10pt,
  innerrightmargin=10pt,
  innertopmargin=8pt,
  innerbottommargin=8pt,
]{warnbox}

% ─── Header / Footer ─────────────────────────────────────────────────────────
\pagestyle{fancy}
\fancyhf{}
\rhead{\small\textcolor{gray}{NIFTY Market Data Engine v3.0}}
\lhead{\small\textcolor{gray}{Data Pipeline Team}}
\rfoot{\small\thepage}
\renewcommand{\headrulewidth}{0.4pt}

% ─── Section formatting ───────────────────────────────────────────────────────
\titleformat{\section}{\large\bfseries\color{codeblue}}{}{0em}{}[\titlerule]
\titleformat{\subsection}{\normalsize\bfseries\color{accent}}{}{0em}{}

% ─── Hyperlink setup ─────────────────────────────────────────────────────────
\hypersetup{
    colorlinks=true,
    linkcolor=accent,
    urlcolor=accent,
    citecolor=accent
}

% ─── Column types ─────────────────────────────────────────────────────────────
\newcolumntype{L}[1]{>{\raggedright\arraybackslash}p{#1}}
\newcolumntype{C}[1]{>{\centering\arraybackslash}p{#1}}

% ──────────────────────────────────────────────────────────────────────────────
\begin{document}

% ─── Title page ───────────────────────────────────────────────────────────────
\begin{titlepage}
\centering
\vspace*{3cm}

{\Huge\bfseries\color{codeblue} NIFTY Market Data Engine}\\[0.5em]
{\LARGE\color{accent} API Documentation}\\[0.3em]
{\large Version 3.0}

\vspace{2cm}
\rule{\linewidth}{1pt}
\vspace{0.5cm}

{\large\bfseries Data Pipeline Team}\\[0.3em]
{\large Mathematical Finance Group Project}

\vspace{0.5cm}
\rule{\linewidth}{1pt}

\vspace{2cm}
{\large January 2026}

\vfill

\begin{infobox}
\textbf{Purpose.} This document is the complete technical reference for the
\texttt{NiftyMarketData} Python class. It is intended for all teams that
consume options market data: Pricing (Team~2b), Hedging, and Volatility
Surface. Data producers on the Data Pipeline Team should also consult this
document to ensure new datasets follow the required schema.
\end{infobox}

\end{titlepage}

% ─── Table of contents ────────────────────────────────────────────────────────
\tableofcontents
\newpage

% ──────────────────────────────────────────────────────────────────────────────
\section{Overview}
% ──────────────────────────────────────────────────────────────────────────────

The NIFTY Market Data Engine provides a clean, professional query interface
over historical NIFTY~50 European options data. Raw data is stored as thousands
of CSV files across monthly folders on a shared Google Drive. This engine
abstracts that entirely: downstream teams write a single Python query and
receive a fully standardised, spot-merged \texttt{pandas} DataFrame.

\subsection{Why This Engine is Needed}

Pricing, Hedging, and Volatility Surface teams require market data as direct
input to:

\begin{itemize}[noitemsep]
    \item Black--Scholes theoretical pricing
    \item Implied volatility computation $\sigma(K,T)$
    \item Greeks computation: $\Delta,\,\Gamma,\,\mathcal{V},\,\Theta$
    \item Volatility surface construction $\sigma(K,T)$
\end{itemize}

Without this engine, each team would need to locate files manually, handle
timestamp reconstruction, and merge spot data themselves --- a slow,
error-prone process. The engine encapsulates all of that.

\subsection{Multi-Year Support}

Version~3.0 extends the engine to support multiple calendar years
(2024, 2025, 2026 and beyond). The correct year folder is determined
automatically from the trade date in each query.

% ──────────────────────────────────────────────────────────────────────────────
\section{Dataset Structure}
% ──────────────────────────────────────────────────────────────────────────────

\subsection{Folder Layout}

All data resides under a single root directory (\texttt{base\_dir}), which is
the shared Google Drive folder. The internal structure must follow this layout
exactly:

\begin{lstlisting}[language={}, numbers=none, basicstyle=\ttfamily\small, backgroundcolor=\color{codegray}, frame=single, rulecolor=\color{gray!40}]
base_dir/
    2024/
        2024JAN/          <- option CSV files for January 2024
        2024FEB/
        ...
        2024DEC/
        2024Nifty/        <- spot index CSV files
    2025/
        2025JAN/
        ...
        2025Nifty/
    2026/
        ...
\end{lstlisting}

\subsection{Option File Naming Convention}

Each option file follows the pattern:

\[
\texttt{NIFTY-\{EXPIRY\}-\{TRADEDATE\}.csv}
\]

Example: \texttt{NIFTY-01FEB24-01JAN24.csv}

\begin{center}
\begin{tabular}{@{}L{3.5cm} L{9cm}@{}}
\toprule
\textbf{Component} & \textbf{Description} \\
\midrule
\texttt{EXPIRY}    & Expiry date in \texttt{DDMMMYY} format, e.g.\ \texttt{01FEB24} \\
\texttt{TRADEDATE} & The date trading occurred, e.g.\ \texttt{01JAN24} \\
\bottomrule
\end{tabular}
\end{center}

\subsection{Raw File Columns}

Each option CSV contains intraday one-minute OHLC bars:

\begin{center}
\begin{tabular}{@{}L{3.5cm} L{9cm}@{}}
\toprule
\textbf{Column} & \textbf{Description} \\
\midrule
\texttt{datetime}       & Intraday time (HH:MM:SS) \\
\texttt{strike\_price}  & Strike price (integer) \\
\texttt{right}          & Option type: \texttt{CE} (Call) or \texttt{PE} (Put) \\
\texttt{open}           & Opening price of 1-minute bar \\
\texttt{high}           & High price \\
\texttt{low}            & Low price \\
\texttt{close}          & Closing price \\
\texttt{volume}         & Contracts traded in this minute \\
\texttt{open\_interest} & Open interest \\
\bottomrule
\end{tabular}
\end{center}

\subsection{Spot File Naming Convention}

\[
\texttt{Nifty-\{YEAR\}\{MONTH\}.csv}
\]

Example: \texttt{Nifty-2024JAN.csv}

Spot files contain columns \texttt{datetime} and \texttt{close}, where
\texttt{close} is the NIFTY~50 spot index level.

% ──────────────────────────────────────────────────────────────────────────────
\section{Key Assumptions}
% ──────────────────────────────────────────────────────────────────────────────

\begin{warnbox}
All consumers of this API must be aware of the following assumptions that
affect data quality and pricing calculations.
\end{warnbox}

\begin{enumerate}
    \item \textbf{No Bid/Ask Data.} This dataset does not contain bid or ask
    quotes. Option market price is proxied by the close price of each
    one-minute bar:
    \[
    P_{\text{market}} = P_{\text{close}}
    \]

    \item \textbf{Zero-Volume Rows.} A large fraction of rows have
    \texttt{volume}~=~0. These represent stale quotes where no trade occurred
    in that minute. For implied volatility calculations, apply
    \texttt{min\_volume~$\geq$~10} to ensure only actively traded contracts
    are used.

    \item \textbf{Spot Price Matching.} Spot prices are merged using
    nearest-timestamp matching (one-minute resolution):
    \[
    S_t \approx S_{\hat{t}}, \quad \hat{t} = \arg\min_{\tau} |t - \tau|
    \]

    \item \textbf{Trading Hours.} NIFTY trades from 09:15 to 15:30~IST.
    Time window queries outside these hours will return empty results.

    \item \textbf{European Options.} All NIFTY index options are European-style.
    This is assumed in all downstream pricing models.
\end{enumerate}

% ──────────────────────────────────────────────────────────────────────────────
\section{Installation and Setup}
% ──────────────────────────────────────────────────────────────────────────────

\subsection{Requirements}

\begin{lstlisting}
pip install pandas
\end{lstlisting}

Python 3.8 or higher is required.

\subsection{File Layout}

Place the engine file inside an \texttt{api/} folder at your project root:

\begin{lstlisting}[language={}, numbers=none, basicstyle=\ttfamily\small, backgroundcolor=\color{codegray}, frame=single, rulecolor=\color{gray!40}]
your_project/
    api/
        __init__.py       <- empty file, required
        marketdatav2.py   <- the engine
    your_notebook.ipynb
\end{lstlisting}

The \texttt{\_\_init\_\_.py} file may be empty; it is required to make
\texttt{api} a Python package.

\subsection{Import and Initialise}

\begin{lstlisting}
from api.marketdatav2 import NiftyMarketData

BASE_DIR = r"G:/SharedDrive/NiftyHistorical"
md = NiftyMarketData(base_dir=BASE_DIR)
\end{lstlisting}

% ──────────────────────────────────────────────────────────────────────────────
\section{Date Format Reference}
% ──────────────────────────────────────────────────────────────────────────────

\begin{center}
\begin{tabular}{@{}L{4cm} L{4cm} L{5.5cm}@{}}
\toprule
\textbf{Parameter} & \textbf{Format} & \textbf{Examples} \\
\midrule
\texttt{expiry} & \texttt{DDMMMYY} & \texttt{01FEB24}, \texttt{27MAR25} \\
\texttt{trade\_date} & \texttt{DDMMMYY} & \texttt{01JAN24}, \texttt{15JAN25} \\
\texttt{start}, \texttt{end} & \texttt{YYYY-MM-DD HH:MM} & \texttt{2024-01-01 10:00} \\
\texttt{timestamp} & \texttt{YYYY-MM-DD HH:MM} & \texttt{2024-01-01 10:00} \\
\texttt{snapshot\_time} & \texttt{HH:MM} & \texttt{10:00}, \texttt{09:30} \\
\bottomrule
\end{tabular}
\end{center}

% ──────────────────────────────────────────────────────────────────────────────
\section{Complete API Reference}
% ──────────────────────────────────────────────────────────────────────────────

\subsection{\texttt{query\_options()} --- Core Query}

The primary method for all option data access. All filters are optional;
omitting them returns the full unfiltered dataset for that expiry and date.

\begin{lstlisting}
df = md.query_options(
    expiry="01FEB24",          # required
    trade_date="01JAN24",      # required
    strikes=[21500, 21700],    # optional: list of int
    option_type="C",           # optional: "C" or "P"
    start="2024-01-01 10:00",  # optional
    end="2024-01-01 11:00",    # optional
    min_volume=10,             # optional: int >= 0
    raise_if_empty=False       # optional: bool
)
\end{lstlisting}

\textbf{Output columns:}

\begin{center}
\begin{tabular}{@{}L{4cm} L{8.5cm}@{}}
\toprule
\textbf{Column} & \textbf{Description} \\
\midrule
\texttt{timestamp}       & Full datetime of the record \\
\texttt{expiry\_date}    & Expiry date (Python \texttt{date} object) \\
\texttt{days\_to\_expiry}& Calendar days until expiry from trade date \\
\texttt{strike}          & Strike price (integer) \\
\texttt{option\_type}    & \texttt{"C"} (Call) or \texttt{"P"} (Put) \\
\texttt{open\_price}     & Open price of the 1-minute bar \\
\texttt{high\_price}     & High price \\
\texttt{low\_price}      & Low price \\
\texttt{close\_price}    & Close price \\
\texttt{market\_price}   & Pricing proxy $= P_{\text{close}}$ \\
\texttt{volume}          & Contracts traded in that minute \\
\texttt{open\_interest}  & Open interest \\
\texttt{spot\_price}     & NIFTY~50 spot level (auto-merged) \\
\bottomrule
\end{tabular}
\end{center}

\subsection{\texttt{list\_expiries()} --- Discovery}

\begin{lstlisting}
expiries = md.list_expiries(trade_date="01JAN24")
# Returns: ['01FEB24', '04JAN24', '11JAN24', '25JAN24', ...]
\end{lstlisting}

Returns all expiry files that exist on disk for the given trade date.
Use before \texttt{query\_options()} to confirm an expiry is available.

\subsection{\texttt{list\_strikes()} --- Discovery}

\begin{lstlisting}
strikes = md.list_strikes(expiry="01FEB24", trade_date="01JAN24")
# Returns: [19850, 20000, 21000, 21100, ..., 23500]
\end{lstlisting}

Returns all strike prices that appear in a given expiry/date file.

\subsection{\texttt{list\_trading\_days()} --- Discovery}

\begin{lstlisting}
days = md.list_trading_days(year=2024, month="JAN")
# Returns: ['01JAN24', '02JAN24', '03JAN24', ...]
\end{lstlisting}

Returns all trading days for which option files exist in a given month.

\subsection{\texttt{get\_atm\_strikes()} --- ATM Grid}

Generates a symmetric grid of strikes centred on the at-the-money level.

\[
K_{\text{ATM}} = \left\lfloor \frac{S_0}{\Delta} \right\rceil \cdot \Delta
\]

where $S_0$ is the opening spot price and $\Delta$ is the step size.

\begin{lstlisting}
atm, grid = md.get_atm_strikes(
    expiry="01FEB24",
    trade_date="01JAN24",
    n_strikes=10,   # strikes each side of ATM
    step=100        # spacing between strikes
)
print(atm)    # 21700
print(grid)   # [21200, 21300, ..., 21700, ..., 22200]
\end{lstlisting}

Total grid size = $2 \times \texttt{n\_strikes} + 1$.

\subsection{\texttt{surface\_snapshot()} --- Volatility Surface}

The primary deliverable for Team~2b. Constructs the complete
$(K,\,T)$ input grid required for implied volatility surface fitting at a
single point in time.

\begin{lstlisting}
surface = md.surface_snapshot(
    trade_date="01JAN24",
    timestamp="2024-01-01 10:00",
    n_expiries=8,       # max expiries to include
    n_strikes=10,       # strikes each side of ATM per expiry
    step=100,
    option_type="C",    # "C", "P", or None
    min_volume=0
)
\end{lstlisting}

\textbf{Use for Black--Scholes IV fitting:}

\begin{lstlisting}
S     = surface["spot_price"]
K     = surface["strike"]
T     = surface["days_to_expiry"] / 365.0   # years
C_mkt = surface["market_price"]

# Solve: C_BS(S, K, T, r, sigma) = C_mkt  for sigma
\end{lstlisting}

\subsection{\texttt{query\_time\_series()} --- Multi-Day Evolution}

Queries the same expiry across multiple trade dates. Useful for studying
time decay and rolling IV across a trading week or month.

\begin{lstlisting}
df_ts = md.query_time_series(
    expiry="01FEB24",
    trade_dates=["01JAN24", "02JAN24", "03JAN24"],
    strikes=[21700],
    option_type="C",
    snapshot_time="10:00",   # optional: fix the intraday time
    min_volume=0
)
\end{lstlisting}

The output contains an additional \texttt{trade\_date} column prepended.

\subsection{\texttt{clear\_spot\_cache()} and \texttt{cache\_status()}}

\begin{lstlisting}
# Inspect what is currently cached
print(md.cache_status())
# {'2024JAN': 375, '2024FEB': 391}

# Clear cache (free memory or after dataset update)
md.clear_spot_cache()
\end{lstlisting}

% ──────────────────────────────────────────────────────────────────────────────
\section{Error Handling}
% ──────────────────────────────────────────────────────────────────────────────

The engine raises typed exceptions with descriptive messages:

\begin{center}
\begin{tabular}{@{}L{4.5cm} L{8cm}@{}}
\toprule
\textbf{Exception} & \textbf{When Raised} \\
\midrule
\texttt{FileNotAvailable}  & Required CSV file not found on disk \\
\texttt{NoDataReturned}    & Query matched 0 rows (\texttt{raise\_if\_empty=True}) \\
\texttt{InvalidParameter}  & Bad value for \texttt{option\_type}, dates, or \texttt{strikes} \\
\texttt{MarketDataError}   & Base class; all engine errors inherit from this \\
\bottomrule
\end{tabular}
\end{center}

Example:

\begin{lstlisting}
from api.marketdatav2 import FileNotAvailable, InvalidParameter, NoDataReturned

try:
    df = md.query_options(expiry="01FEB24", trade_date="01JAN24",
                          min_volume=999999, raise_if_empty=True)
except NoDataReturned as e:
    print(e)   # prints actionable guidance
\end{lstlisting}

% ──────────────────────────────────────────────────────────────────────────────
\section{Typical Workflow: Pricing Team}
% ──────────────────────────────────────────────────────────────────────────────

The following is the recommended end-to-end workflow for Team~2b:

\begin{lstlisting}
from api.marketdatav2 import NiftyMarketData

md = NiftyMarketData(base_dir=BASE_DIR)

# Step 1 — Confirm available expiries
expiries = md.list_expiries("01JAN24")

# Step 2 — Get surface snapshot
surface = md.surface_snapshot(
    trade_date="01JAN24",
    timestamp="2024-01-01 10:00",
    n_expiries=8,
    n_strikes=10,
    option_type="C",
    min_volume=0
)

# Step 3 — Extract inputs
S     = surface["spot_price"].values
K     = surface["strike"].values
T     = surface["days_to_expiry"].values / 365.0
C_mkt = surface["market_price"].values

# Step 4 — Implied volatility solve
# sigma = your_iv_solver(S, K, T, r, C_mkt, "C")
\end{lstlisting}

% ──────────────────────────────────────────────────────────────────────────────
\section{Summary of All Methods}
% ──────────────────────────────────────────────────────────────────────────────

\begin{center}
\begin{tabular}{@{}L{5cm} L{8.5cm}@{}}
\toprule
\textbf{Method} & \textbf{Purpose} \\
\midrule
\texttt{query\_options(...)}         & Core data access with all filters \\
\texttt{list\_expiries(trade\_date)} & Discover expiries on a date \\
\texttt{list\_strikes(expiry, date)} & Discover strikes in a file \\
\texttt{list\_trading\_days(y, m)}   & Discover trading days in a month \\
\texttt{get\_atm\_strikes(...)}      & Build ATM-centred strike grid \\
\texttt{query\_time\_series(...)}    & Multi-day evolution query \\
\texttt{surface\_snapshot(...)}      & Full vol-surface input grid \\
\texttt{clear\_spot\_cache()}        & Free memory / reset cache \\
\texttt{cache\_status()}             & Inspect cached months \\
\bottomrule
\end{tabular}
\end{center}

% ──────────────────────────────────────────────────────────────────────────────
\section{Contact and Maintenance}
% ──────────────────────────────────────────────────────────────────────────────

This engine is maintained by the \textbf{Data Pipeline Team}. For the following
situations, contact the Data Pipeline Team rather than modifying files directly:

\begin{itemize}[noitemsep]
    \item A \texttt{FileNotAvailable} error for a date that should have data.
    \item New data for 2025 or 2026 not yet loaded.
    \item Schema changes in the raw CSV format.
    \item Questions about risk-free rate or dividend yield integration.
\end{itemize}

\textbf{Do not modify the raw CSV files or folder structure.} The engine
depends on the naming conventions being exact.

\vspace{1em}
\hrule
\vspace{0.5em}
{\small Data Pipeline Team $\cdot$ Mathematical Finance Group Project $\cdot$
Version 3.0 $\cdot$ January 2026}

\end{document}
